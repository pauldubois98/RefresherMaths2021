\documentclass[11pt,a4paper]{report}

%format
\usepackage[utf8]{inputenc}
\usepackage[T1]{fontenc}
\usepackage[english]{babel}
\usepackage[margin=2.5cm]{geometry}
%math
\usepackage{amsthm}
\usepackage{amsmath}
\usepackage{amsfonts}
\usepackage{amssymb}
\usepackage{stmaryrd}
\usepackage{nicefrac}
%others
\usepackage{hyperref}
\usepackage{graphicx}

%environments
\newtheorem*{remark}{Remark}
\newtheorem*{notation}{Notation}
\newtheorem*{definition}{Definition}
\newtheorem*{question}{Question}
\newtheorem*{proposition}{Proposition}
\newtheorem*{property}{Property}
\newtheorem{lemma}{Lemma}[section]
\newtheorem{theorem}{Theorem}[section]
\newtheorem{corollary}{Corollary}[section]
\newtheorem{conjecture}{Conjecture}[section]
%commands
%\newcommand{\name}[num]{definition}
\newcommand{\primes}{\mathbb{P}}
%\newcommand{\P}{\mathbb{P}}
\newcommand{\N}{\mathbb{N}}
\newcommand{\Z}{\mathbb{Z}}
\newcommand{\Q}{\mathbb{Q}}
\newcommand{\D}{\mathbb{D}}
\newcommand{\R}{\mathbb{R}}
\newcommand{\C}{\mathbb{C}}
\newcommand{\F}{\mathbb{F}}
\newcommand{\B}{\mathbb{B}}
\newcommand{\Norm}[2][]{\text{Norm}_{#1}(#2)}
\newcommand{\inner}[2]{\left\langle #1,#2 \right\rangle}
\newcommand{\floor}[1]{\lfloor #1 \rfloor}
\newcommand{\ceil}[1]{\lceil #1 \rceil}
\newcommand{\abs}[1]{| #1 |}
\newcommand{\card}[1]{| #1 |}
\newcommand{\curt}[1]{\sqrt[3]{#1}}
\newcommand{\Ker}[1]{\text{Ker}(#1)}
\newcommand{\Image}[1]{\text{Im}(#1)}
\newcommand{\Trace}[1]{\text{Tr}(#1)}
\newcommand{\Det}[1]{\text{Det}(#1)}
\newcommand{\degree}[1]{\partial #1}
\newcommand{\Pow}[1]{\mathcal{P}(#1)}


\title{Refresher Math Course}
\author{Paul Dubois}
\date{September 2021}



\begin{document}
	\maketitle
	
	\begin{abstract}
		This course teaches basic mathematical methodologies for proofs.
		It is intended for students with a lack of mathematical background, or with a lack of confidence in mathematics.
		The course will try to cover most of the prerequisites of the courses in the Master, mainly linear algebra, differential calculus, integration, and asymptotic analysis.
	\end{abstract}

	\tableofcontents
	\newpage
	
	\section*{Introduction}
	\paragraph{Presentation}
	\begin{itemize}
		\item Paul Dubois
		\item will be teaching this refresher math course
		\item email (for any question), answer within 1 working day
	\end{itemize}
	\paragraph{Course Format}
	\subparagraph{Lectures}
	\begin{itemize}
		\item 8*3h
		\item 1h20min lecture - $\nicefrac{1}{3}$h break - 1h20min lecture
		\item No pb class planned, but lectures will have integrated live exercises
		\item Interrupt if needed (but may also ask at the end of the lecture)
		\item Lectures are recorded (if ever needed)
		\item 1st lecture ever => too fast/too slow: let me know
		\item May assume you know a concept/notation that you have never heard of, let me know if this happens
	\end{itemize}
	\subparagraph{Examination}
	\begin{itemize}
		\item The course is pass/fail
		\item Most (in fact hopefully all) of you will pass
		\item There will be a full exercise sheet per lecture, it is advised to attempt it all (only one will be compulsory).
		\item Hand-in 1 exercise per lecture (i.e., 8 in total), due 2 weeks after the lecture
		\item Best $\nicefrac{(n-1)}{n}$ count (i.e., best $\nicefrac{7}{8}$ in our case), need avg $\geq 50 \%$ to pass
		\item In the unlikely event of not passing, will be able to do an extra work
	\end{itemize}
	\paragraph{Questions?}
	
	\newpage
	
	
	\include{1-SetsAndLogic}
	\chapter{Proofs methods}

\section{Direct implication}
Want to show $A$: may show $B$ and $B \implies A$, or $C$ and $C \implies B$ and $B \implies A$.

\section{Case dis-junction}
Split in cases.

E.g.: show $n$ and $n^2$ have the same parity (take $n$ odd then $n$ even).

\section{Contradiction}
Suppose the opposite, derive a contradiction (i.e. $A$ and $\not A$) and conclude.

E.g.: show $\sqrt{2} \not\in \Q$ (suppose $\sqrt{2}=\nicefrac{a}{b}$, WLOG $a,b \in \N$ co-prime).

\section{Induction}
Want to show $P_n$ for $n \geq n_0$: show $P_n \implies P_{n+1}$ and $P_{n_0}$.

E.g.: show $\sum_{k=0}^{n} k = \frac{n(n+1)}{2}$ for all $n \in \N$.

\section{Existence and Uniqueness}
It is common to show existence and/or uniqueness.

E.g.: Existence and uniqueness in Euclidean division: 
$$\forall a \in \Z, b \in \N^*, \exists ! \ q \in \Z, r \in \left[ 0, b \right[ \cap \N \text{ s.t. } a=bq+r$$
Use $q = \max\{ k \in \N \mid bk \leq a \}$, $r = a-bq$.

\begin{question}
	\begin{itemize}
		\item Show that $n$ divisible by 6 if and only if $n$ divisible by 2 and 3.
		\item Show $\sqrt{3} \not\in \Q$.\footnote{See \url{https://en.wikipedia.org/wiki/Modular_arithmetic} and use it to show $3 \mid n^2 \implies 3 \mid n$.}
		\item Show that $12n-6$ is divisible by 6 for every positive integer $n$.
		\item Show that $2^n \geq 2n$ for all $n \in \N$
	\end{itemize}
\end{question}
	\chapter{Functions Properties}
$$f: X \to Y \quad A \subseteq X, B \subseteq Y$$
\begin{definition}[Image]
	$f(A) = \{ y \in Y \mid \exists x \in A \text{ s.t. } f(x)=y \}$
\end{definition}
\begin{definition}[Inverse Image]
	$f^{-1}(B) = \{ x \in X \mid f(x) \in B \}$
\end{definition}
\begin{definition}[Fiber]
	Fiber of $y$ is inverse image of $\{y\}$.
\end{definition}
\begin{definition}[Well definedness]
	$\forall x \in X, \exists ! y \in Y \text{ s.t. } f(x) = y$\\
	%	-- or --\\
	%	$\forall x \in X, \exists y \in Y \text{ s.t. } f(x) = y$\\
	%	$\forall y,y' \in Y, y \neq y', f^{-1}(y) \cap f^{-1}(y') = \emptyset$
\end{definition}
\begin{definition}[Injectivity]
	$\forall x,x' \in X, x \neq x', f(x) \neq f(x')$
\end{definition}
\begin{definition}[Surjectivity]
	$\forall y \in Y, \exists x \in X \text{ s.t. } f(x) = y$
\end{definition}
\begin{definition}[Bijectivity]
	Injectivity plus Surjectivity:
	$\forall y \in Y, \exists! x \in X \text{ s.t. } f(x) = y$
\end{definition}
\begin{definition}[Invertibility]
	$f^{-1}: Y \to X$ well defined.
\end{definition}
\begin{remark}[Alternative Definition of Inverse]
	$f \circ f^{-1} = Id \mid_X$  and $f^{-1} \circ f = Id \mid_Y$
\end{remark}
\begin{remark}[Invertibility and Bijectivity]
	$f$ bijective $\iff$ $f$ invertible.
\end{remark}
\begin{remark}[Inverse is Invertible]
	$f^{-1}$ is invertible, and $(f^{-1})^{-1}=f$.
\end{remark}
\begin{property}[Injection between finite intervals]
	$n,p \in \N^*$, there exists an injection $f:\left[ 1;n \right] \rightarrow \left[ 1;p \right]$ if and only if $n\leq p$.
	%$n,p \in \N^*$, there exists an injection $f:\llbracket 1;n \rrbracket \rightarrow \llbracket 1;p \rrbracket$ if and only if $n\leq p$.
\end{property}



	\chapter{Finite Cardinalities}

\begin{definition}[Cardinality] For finite sets:\\
	\emph{\underline{Intuitively}:} $\card{X} = n \in \N$ if there are $n$ elements in the set.\\
	\emph{\underline{Mathematically}:} $\card{X} = n \in \N$ if there is a bijection between $X$ and $\llbracket 1,n \rrbracket$.
\end{definition}
\begin{property}[Cardinality of Disjoints]
	$X,Y$ disjoint sets: $\card{X \cup Y} = \card{X} + \card{Y}$\\
	Extension: $X_1, \dots, X_n$ pairwise disjoint sets (i.e. $X_i \cap X_j = \emptyset \ \forall i \neq j$): $\card{\bigcup_{k=1}^n X_k} = \sum_{k=1}^{n} \card{X_k}$
\end{property}
\begin{proof}
	Shift bijection of $Y$ by $\card{Y}$; use induction.
\end{proof}
\begin{property}[Cardinality of Complement]
	$X \subseteq Y$: $\card{Y \setminus X} = \card{Y} - \card{X}$
\end{property}
\begin{proof}
	Use previous property with $X$ \& $Y \setminus X$ disjoint.
\end{proof}
\begin{property}[Cardinality of Cartesian Products]
	$X,Y$ sets: $\card{X \times Y} = \card{X} * \card{Y}$\\
	Extension: $X_1, \dots, X_n$ sets: $\card{\prod_{k=1}^n X_k} = \prod_{k=1}^{n} \card{X_k}$
	%proof: 
\end{property}
\begin{proof}
	$X \times \{y_k\}$  are all disjoint for $k \in \llbracket 1,\card{Y} \rrbracket$; use induction.
\end{proof}
\begin{property}[Cardinality of Sub-list]
	$X$ sets: $\card{ \{ Y \text{ list} \mid \card{Y}=n \text{ and } y \in Y \implies y \in X \} } = \card{X}^n$
\end{property}
\begin{proof}
	Just count!
\end{proof}
\begin{property}[Cardinality of Ordered Subsets]
	$X$ sets: $\card{ \{ Y \text{ ordered set} \mid \card{Y}=n \text{ and } y \in Y \implies y \in X \} } = \frac{\card{X}!}{(\card{x}-n)!}$
\end{property}
\begin{proof}
Just count!
\end{proof}
\begin{property}[Cardinality of Subsets]
	$X$ sets: $\card{ \{ Y \subseteq X \mid \card{Y}=n \} } = \binom{\card{X}}{n}$
\end{property}
\begin{proof}
	Just count!
\end{proof}
\begin{property}[Cardinality of Sets of Functions]
	%Similar to Cartesian product:\\
	$\card{ \{f: X \to Y\} } = \card{Y}^{\card{X}}$
\end{property}
\begin{proof}
	Just count!
\end{proof}
\begin{property}[Cardinality of Sets of Injections]
	$\card{ \{f: X \to Y \mid f \text{ injective} \} } = \frac{\card{Y}!}{(\card{Y}-\card{X})!}$
\end{property}
\begin{proof}
	Count (without repetition).
\end{proof}
\begin{property}[Cardinality of Sets of Surjections]
	$\card{ \{f: X \to Y \mid f \text{ surjective} \} } = \card{Y}^{\card{X}} - \card{Y}*(\card{Y}-1)^{\card{X}}$
\end{property}
\begin{proof}
	All functions but the non surjective ones.
\end{proof}
\begin{property}[Cardinality of Sets of Bijections]
	$\card{ \{f: X \to Y \mid f \text{ bijective} \} } = \card{Y}! = \card{X}!$
\end{property}
\begin{proof}
	Bijection is an injection between two sets of the same size.
\end{proof}

\begin{question}
	\begin{itemize}
		\item For $n$ students, if we record the order of people getting out of the room, how many possibilities are there?
		\item Bench for 10 people, we have 5 boys, 5 girls, how many arrangements are there such that two boys/two girls are never seated next to each others?
		\item Bench for 11 people, we have 6 boys, 5 girls, how many arrangements are there such that two boys/two girls are never seated next to each others?
	\end{itemize}
\end{question}



	\chapter{Infinite Cardinalities}
\begin{definition}[Alphabet]
	$\mathcal{A} = \{ a,b,c, \dots, z \}$
\end{definition}
To compare the size of infinite sets, we use bijections, injections:
\begin{definition}[Comparing Sets]
	$f: X \to Y \text{ injective} \implies \card{X} \leq \card{Y}$
	$f: X \to Y \text{ surjective} \implies \card{X} \geq \card{Y}$
	$f: X \to Y \text{ bijective} \implies \card{X} = \card{Y}$
\end{definition}
Note that together with $\card{\left[ 1,n \right]}=n$, this defines cardinality.
\begin{definition}[Countable sets]
	A set is countable if it has the same cardinality as the naturals (i.e. $X$ is countable if $\card{X} = \card{\N}$).
\end{definition}

\begin{property}[Countable Union Finite]
	$\card{\N \cup \mathcal{A}} = \card{\N}$
\end{property}
\begin{property}[Countable Union Countable / Integers]
	$\card{\Z} = \card{\N \cup \N^*} = \card{\N}$
\end{property}
\begin{property}[Countable Union of Finites]
	$\card{X_n}<\infty \ \forall n \in \N \implies \card{\bigcup_{n \in \N} X_n} = \card{\N}$
	%proof: stack the X_n on \N
\end{property}
\begin{property}[Countable Union of Countables / Rationals]
	$\card{\Q} = \card{\bigcup_{n \in \N^*} \{ \nicefrac{m}{n} \mid m \in \Z \}} = \card{\N}$
\end{property}
\begin{property}[Power set of Countables / Reals]
	$\card{ \left[ 0,1 \right[ } = \card{\Pow{\N}} = \card{\{0,1\}^{\N}} > \card{\N}$
\end{property}






	\chapter{Spaces}
Mathematical Space: Object based on a set with more structure.
\section{Metric Space}
A metric space is a set $X$ together with a metric distance $d: X \times X \to \R^+$.\\
$d$ is a metric if it satisfies the following axioms:
\begin{itemize}
	\item Non-degenerative: $d(x,y)=0 \iff x=y$
	\item Symmetric: $d(x,y) = d(y,x)$
	\item Triangle inequality: $d(x,z) \leq d(x,y) + d(y,z)$
\end{itemize}

\section{Norm Space}
A norm space is a set $X$ together with a norm $\abs{\_}: X \to \R^+$.\\
$\abs{\_}$ is a norm if it satisfies the following axioms:
\begin{itemize}
	\item Non-degenerative: $\abs{x}=0 \iff x=0$
	\item Homogeneity: $\abs{\lambda x} = \lambda \abs{x} \qquad \lambda \in \R^+$
	\item Triangle inequality: $\abs{x+y} \leq \abs{x} + \abs{y}$
\end{itemize}
\begin{property}[Norm Implies Metric]
	Letting $d(x,y) = \abs{x-y}$.
\end{property}

\section{Inner Product Space}
An inner product space is a set $X$ together with an inner product $\inner{\_}{\_}: X \times X \to \C$.\\
$\inner{\_}{\_}$ is an inner product if it satisfies the following axioms:
\begin{itemize}
	\item Linear (in $1^{\text{st}}$ argument): $\inner{\lambda x}{y} = \lambda \inner{x}{y} \quad \lambda \in \C$ and $\inner{x+x'}{y} = \inner{x}{y} +\inner{x'}{y}$
	\item Conjugate symmetry: $\abs{x+y} \leq \abs{x} + \abs{y}$
	\item Positive definiteness $\inner{x}{x}>0 \ \forall x \neq 0$
	\item \textit{(implied)} Non-degenerative: $\inner{x}{0}=0$ and $\inner{0}{x}=0$
	\item \textit{(implied)} Conjugate linear (in $2^{\text{nd}}$ argument): $\inner{x}{\lambda y} = \bar{\lambda} \inner{x}{y} \quad \lambda \in \C$ and $\inner{x}{y+y'} = \inner{x}{y} +\inner{x}{y'}$
\end{itemize}
\begin{property}[Inner Product implies Norm]
	Letting $\abs{x} = \sqrt{\inner{x}{x}}$.
\end{property}
\begin{definition}[Orthogonal / Normal]
	$x,y \text{ orthogonal } \iff x \perp y \iff \inner{x}{y}=0$
\end{definition}
\begin{property}[Pythagoras Theorem]
	$x \perp y \implies \abs{x+y}^2 = \abs{x}^2 + \abs{y}^2$
\end{property}
\begin{property}[Parallelogram Identity]
	$\abs{x+y}^2 +\abs{x-y}^2 = 2 (\abs{x}^2 + \abs{y}^2)$
\end{property}
\begin{property}[Polarization Identity]
	$4 \inner{x}{y} = \abs{x+y}^2 - \abs{x-y}^2 + i(\abs{x+iy}^2 - \abs{x-iy}^2)$
\end{property}

\begin{question}
	Draw ball of radius one in $\R^2$ for the following norms: $\abs{\_}_1$, $\abs{\_}_2$, $\abs{\_}_3$, $\abs{\_}_{\infty}$.
\end{question}




	
	
	
\end{document}